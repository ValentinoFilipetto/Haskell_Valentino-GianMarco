\documentclass[12pt,a4paper]{article}
\input{latexmacros.tex}

\title{My Report}
\author{Me}
\date{\today}
\hypersetup{pdfauthor={Me}, pdftitle={My Report}}

\begin{document}

\maketitle

\begin{abstract}
We give a toy example of a report in \emph{literate programming} style.
The main advantage of this is that source code and documentation can
be written and presented next to each other.
We use the listings package to typeset Haskell source code nicely.
\end{abstract}

\vfill

\tableofcontents

\clearpage

% We include one file for each section. The ones containing code should
% be called something.lhs and also mentioned in the .cabal file.

\input{Howto.tex}

\input{lib/Main.lhs}

\input{test/simpletests.lhs}

\input{Profiling.tex}

\section{Conclusions}

The project includes a prover for the implication-free fragment of intuitionistic propositional logic as well as a prover for the implication free fragment of classical propositional logic. The impossibility of testing whether formulas containing implications are intuitionistically valid is an obvious shortcoming of the project, and it is the first thing that comes to mind when thinking about directions to expand this work. This should be possible without changing the structure of the program too much. On the other hand, it is well known that classical implication is definable in terms of the other connectives, therefore the classical prover can actually be used as a prover for all of classical propositional logic. Since this prover is obtained by a slight modification of an intuitionistic prover, our conjecture is that it is way less performant than most provers developed specifically for classical logic, whereas it is not known to us whether other existing provers for intuitionistic propositional logic perform much better than ours.


\addcontentsline{toc}{section}{Bibliography}
\bibliographystyle{alpha}
\bibliography{references.bib}

\end{document}
